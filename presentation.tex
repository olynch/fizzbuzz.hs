\documentclass{beamer}
\usetheme{default}

\usepackage{minted}

\title{FizzBuzz in Haskell}
\author{Owen Lynch}

\newminted{haskell}{fontsize=\large, 
                    numbersep=8pt,
                    gobble=2,
                    bgcolor=bg} 
               		

\begin{document}

\definecolor{bg}{rgb}{0.95,0.95,0.95}

\begin{frame}
  \titlepage
\end{frame}

\defverbatim[colored]\factCode{
\begin{haskellcode}
  fact :: Int -> Int
  fact 0 = 1
  fact n = n * fact (n-1)
\end{haskellcode}
}

\begin{frame}{Introducing Haskell}
  \factCode
  \begin{enumerate}
    \pause
    \item Types
    \pause
    \item Functions
    \pause
    \item Pattern Matching
    \pause
    \item Recursion
  \end{enumerate}
\end{frame}

\begin{frame}{Myths about Haskell}
  \begin{enumerate}
    \pause
    \item Haskell is hard
    \pause
    \begin{enumerate}
      \item \textit{Learning} Haskell is hard, \textit{programming} Haskell is easy!
      \pause
      \item Haskell lets you choose the level of abstraction that you are comfortable with, and you can slowly increase this while being productive the whole time
    \end{enumerate}
    \pause
    \item Haskell is for techbro superprogrammers
    \begin{enumerate}
      \pause
      \item Haskell is for people who can't keep lots of stuff in their head at once and want the compiler to make sure they aren't messing things up
      \pause
      \item The Haskell community is one of the nicest and friendliest online communities I've been a part of, in part because there is so much to learn that everyone is comparatively a noob
    \end{enumerate}
    \pause
    \item Haskell is for academic ivory-towerists who do too much category theory for their own good
    \begin{enumerate}
      \pause
      \item Haskell is actually a very practical language for all sorts of tasks, and it has been battletested in industry for decades
    \end{enumerate}
  \end{enumerate}
\end{frame}

\begin{frame}{Introducing FizzBuzz}
  Make a list of the numbers from 1 to 100, except...
  \begin{enumerate}
    \pause
    \item For every number divisible by 3, put ``Fizz'' instead
    \pause
    \item For every number divisible by 5, put ``Buzz'' instead
    \pause
    \item For every number divisible by 5 and 3, put ``FizzBuzz'' instead
  \end{enumerate}
  \pause
  How would you solve FizzBuzz?
\end{frame}

\defverbatim[colored]\introLists{
\begin{haskellcode}
  data [a] = [] | a : [a]
  -- data List a = Empty | Cons a (List a)
  -- (:) x xs == x : xs
  -- ([]) a == [a]
\end{haskellcode}
}

\defverbatim[colored]\rangeDef{
\begin{haskellcode}
  range :: Int -> Int -> [Int]
  range n m
    | n == m    = m:[]
    | otherwise = m:(range (n+1) m)
  -- range n m == [n..m]
\end{haskellcode}
}

\defverbatim[colored]\introFunctors{
\begin{haskellcode}
  class Functor f where
    fmap :: (a -> b) -> f a -> f b

  instance Functor ([]) where
    fmap g [] = []
    fmap g (x:xs) = (g x):(fmap g xs)
    -- fmap :: (a -> b) -> ([]) a -> ([]) b
    -- fmap :: (a -> b) -> [a] -> [b]
\end{haskellcode}
}

\defverbatim[colored]\firstSolution{
\begin{haskellcode}
  -- String == [Char]

  fizzbuzz1 :: Int -> String
  fizzbuzz1 n
    | rem n 15 == 0 = "FizzBuzz"
    | rem n  5 == 0 = "Buzz"
    | rem n  3 == 0 = "Fizz"
    | otherwise     = show n

  sol1 :: [String]
  sol1 = fmap fizzbuzz1 [1..100]
\end{haskellcode}
}

\begin{frame}{Lists in Haskell}
  \introLists
  \begin{enumerate}
    \pause
    \item Data declarations
    \pause
    \item Parametric data declarations
    \pause
    \item Special syntax for stuff
  \end{enumerate}
\end{frame}

\defverbatim[colored]\exampleLists{
\begin{haskellcode}
  firstFourPrimes :: [Int]
  firstFourPrimes = 2:(3:(5:(7:[])))

  everFlavoredBeanFlavors :: [String]
  everFlavoredBeanFlavors =
    ["earwax", "vomit", "marmalade", "spinach"]
\end{haskellcode}
}

\begin{frame}{What does it look like?}
  \exampleLists
  (put drawing of a linked list here)
\end{frame}

\begin{frame}{A useful function}
  \rangeDef
  \begin{enumerate}
    \pause
    \item Currying
    \pause
    \item Guard Notation
  \end{enumerate}
\end{frame}

\begin{frame}{Functors! (Scary? No!)}
\introFunctors
\end{frame}

\begin{frame}{What does it look like?}
  (put drawing of a functor diagram here)
\end{frame}

\begin{frame}{Luke, use the fmap!}
\firstSolution  
\end{frame}

\begin{frame}{Discuss first solution}
  \begin{enumerate}
    \item What did we like?
      \begin{enumerate}
        \pause
        \item Types and Typeclasses!
        \pause
        \item Recursion!
        \pause
        \item Combinators that allowed us to not do recursion!
        \pause
        \item Case syntax!
      \end{enumerate}
    \pause
    \item What did we not like?
      \begin{enumerate}
        \pause
        \item Not extensible enough! (What about Bazz??)
      \end{enumerate}
  \end{enumerate}
\end{frame}

\defverbatim[colored]\fizzbuzzbazzOne{
\begin{haskellcode}
  fizzbuzzbazz1 :: Int -> String
  fizzbuzzbazz1 n
    | rem n 105 == 0 = "FizzBuzzBazz"
    | rem n  35 == 0 = "BuzzBazz"
    | rem n  21 == 0 = "FizzBazz"
    | rem n  15 == 0 = "FizzBuzz"
    | rem n   7 == 0 = "Bazz"
    | rem n   5 == 0 = "Buzz"
    | rem n   3 == 0 = "Fizz"
    | otherwise      = show n
\end{haskellcode}
}

\begin{frame}{FizzBuzzBazz}
  \fizzbuzzbazzOne
  \pause
  This is terrible!
\end{frame}

\defverbatim[colored]\introMonoids{
\begin{haskellcode}
  class Monoid m where
    mempty :: m
    mappend :: m -> m -> m
    -- (<>) == mappend
\end{haskellcode}
}

\defverbatim[colored]\identityEq{
\begin{haskellcode}
  a <> mempty == a == mempty <> a
\end{haskellcode}
}

\defverbatim[colored]\associativityEq{
\begin{haskellcode}
  (a <> b) <> c == a <> (b <> c)
\end{haskellcode}
}

\begin{frame}{And now for something completely different}
  \introMonoids
  \pause
  We require a couple laws
  \pause
  \begin{enumerate}
    \item Identity \\
    \pause
    \identityEq
    \pause
    \item Associativity \\
    \pause
    \associativityEq
  \end{enumerate}
\end{frame}

\defverbatim[colored]\listMonoid{
\begin{haskellcode}
  (++) :: [a] -> [a] -> [a]
  [] ++ ys = ys
  (x:xs) ++ ys = x:(xs ++ ys)

  instance Monoid [a] where
    mempty = []
    mappend = (++)
\end{haskellcode}
}

\defverbatim[colored]\maybeMonoid{
\begin{haskellcode}
  data Maybe a = Just a | Nothing

  instance (Monoid a) => Monoid (Maybe a) where
    mempty = Nothing
    mappend Nothing Nothing = Nothing
    mappend (Just a) Nothing = Just a
    mappend Nothing (Just b) = Just b
    mappend (Just a) (Just b) = Just (a <> b)
\end{haskellcode}
}

\begin{frame}{You already know lots of monoids!}
  Take a couple minutes and think of monoids.
\end{frame}

\begin{frame}{List Monoid}
  \listMonoid
\end{frame}

\begin{frame}{Maybe Monoid}
  \maybeMonoid
  \pause
  Note the "type constraint" in the instance for the Maybe monoid; we need this in order to deal with the last case.
\end{frame}

\defverbatim[colored]\usefulFunctions{
\begin{haskellcode}
  foldl :: (a -> b -> a) -> a -> [b] -> a
  foldl f acc [] = acc
  foldl f acc (x:xs) = foldl f (f acc x) xs
  -- sum == foldl (+) 0
  -- length == foldl (\n _ -> n + 1) 0

  mconcat :: (Monoid m) -> [m] -> m
  mconcat = foldl mappend mempty
  -- mconcat ["tweedle", "dee"] == "tweedledee"
\end{haskellcode}
}

\begin{frame}{Two useful functions}
  \usefulFunctions
\end{frame}

\defverbatim[colored]\workingWithMaybes{
\begin{haskellcode}
  fromMaybe :: a -> Maybe a -> a
  fromMaybe def Nothing = def
  fromMaybe _ (Just y) = y

  makeMaybe :: Bool -> a -> Maybe a
  makeMaybe True x = Just x
  makeMaybe False _ = Nothing
\end{haskellcode}
}

\begin{frame}{Working with Maybes}
  \workingWithMaybes
\end{frame}

\defverbatim[colored]\fizzbuzzTwo{
\begin{haskellcode}
  zzer1 :: [(Int, String)] -> Int -> String
  zzer1 zzConds n = fromMaybe (show n) zzs
    where
      zzs = mconcat (fmap maybeZz zzConds)
      maybeZz (k, zz) = makeMaybe (rem n k == 0) zz

  sol2 :: [String]
  sol2 = fmap (zzer1 fizzbuzz) [1..100]
    where fizzbuzz = [(3, "Fizz"), (5, "Buzz")]
\end{haskellcode}
}

\begin{frame}{FizzBuzz revisited}
  \fizzbuzzTwo
\end{frame}

\begin{frame}{Reflections on take 2}
  What do we like?
  \pause
  \begin{enumerate}
    \item Much more extensible!
    \pause
    \item Shorter!
  \end{enumerate}
  \pause
  What do we not like?
  \begin{enumerate}
    \pause
    \item Division
  \end{enumerate}
\end{frame}

\begin{frame}{FizzBuzz: Hard Mode}
  FizzBuzz, but...
  \begin{enumerate}
    \pause
    \item For every prime number, put ``Buzz'' instead
    \pause
    \item For every fibonacci number, put ``Fizz'' instead
    \pause
    \item For every number that is a fibonacci number and a prime number, put ``FizzBuzz'' instead
  \end{enumerate}
\end{frame}

\defverbatim[colored]\infiniteLists{
\begin{haskellcode}
  repeat :: a -> [a]
  repeat x = x:(repeat x)
  -- take 3 (repeat 1) == [1,1,1]

  primes :: [Int]
  primes = from 2
    where
      from k = k:(filter (ndiv k) (from (k+1)))
      ndiv k n = rem n k /= 0 
  -- take 4 primes == [2,3,5,7]
\end{haskellcode}
}

\begin{frame}{Infinity and Beyond!}
 \infiniteLists
 \pause
 \begin{enumerate}
   \item Haskell is lazy -- it only computes values when you ask for them
 \end{enumerate}
\end{frame}

\defverbatim[colored]\zipping{
\begin{haskellcode}
  zipWith :: (a -> b -> c) -> [a] -> [b] -> [c]
  zipWith _ [] _ = []
  zipWith _ _ [] = []
  zipWith f (x:xs) (y:ys) = (f x y):(zip f xs ys)

  fibs :: [Int]
  fibs = 0:1:(zipWith (+) fibs (tail fibs))
  -- This one's a braintwister!
\end{haskellcode}
}

\begin{frame}{Zipping Lists}
  \zipping
\end{frame}

\defverbatim[colored]\fizzbuzzThree{
\begin{haskellcode}
  spacer :: [Int] -> a -> [Maybe a]
  spacer (n:ns) s = loop (n-1) (n:ns)
    where
      loop 0 (k1:k2:ns) = (Just s):(loop (k2-k1-1) (k2:ns))
      loop k ns = Nothing:(loop (k-1) ns)

  zzer2 :: [([Int], String)] -> [Maybe String]
  zzer2 zzConds = foldl (zipWith (<>)) init zzLists
    where
      zzLists = map (uncurry spacer) zzConds
      init = repeat Nothing

  sol3 :: [String]
  sol3 = zipWith fromMaybe (fmap show [1..100]) zzs
    where zzs = zzer2 [(primes,"Fizz"), ((tail (tail fibs)),"Buzz")]
\end{haskellcode}
}

\begin{frame}{FizzBuzz: The Final Showdown}
  \fizzbuzzThree
\end{frame}

\end{document}
